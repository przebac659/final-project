% !TeX spellcheck = en_GB
%%%%%%%%%%%%%%%%%%%%%%%%%%%%%%%%%%%%%%%%%%
%                                        %
%    Engineer thesis LaTeX template      % 
%  compliant with the SZJK regulations   %
%                                        %
%%%%%%%%%%%%%%%%%%%%%%%%%%%%%%%%%%%%%%%%%%
%                                        %
%  (c) Krzysztof Simiński, 2018-2023     %
%                                        %
%%%%%%%%%%%%%%%%%%%%%%%%%%%%%%%%%%%%%%%%%%
%                                        %
% The latest version of the templates is %
% available at                           %
% github.com/ksiminski/polsl-aei-theses  %
%                                        %
%%%%%%%%%%%%%%%%%%%%%%%%%%%%%%%%%%%%%%%%%%
%
%
% This LaTeX project formats the final thesis 
% with compliance to the SZJK regulations. 
% Please to not change formatting (fonts, margins,
% bolds, italics, etc).
%
% You can compile the project in several ways.
%
% 1. pdfLaTeX compilation
% 
% pdflatex main
% bibtex   main
% pdflatex main
% pdflatex main
%
%
% 2. XeLaTeX compilation
%
% Compilation with the XeLaTeX engine inserts Calibri font
% in the title page. Of course the font has to be installed.
% 
% xelatex main
% bibtex  main
% xelatex main
% xelatex main
%
%
%%%%%%%%%%%%%%%%%%%%%%%%%%%%%%%%%%%%%%%%%%%%%%%%%%%%%%%%%%%%%%
% If you have any questions, remarks, just send me an email: %
%            krzysztof.siminski(at)polsl.pl               %
%%%%%%%%%%%%%%%%%%%%%%%%%%%%%%%%%%%%%%%%%%%%%%%%%%%%%%%%%%%%%%

% We would like to improve the LaTeX templates 
% of final theses. By answering the questions 
% in the survey whose address your can find below 
% you help us to do so. The survey is completely 
% anonimous. Thank you!
%
% https://docs.google.com/forms/d/e/1FAIpQLScyllVxNKzKFHfILDfdbwC-jvT8YL0RSTFs-s27UGw9CKn-fQ/viewform?usp=sf_link
%
%%%%%%%%%%%%%%%%%%%%%%%%%%%%%%%%%%%%%%%%%%%%%%%%%%%%%%%%%%%%%%%%%%%%%%%%%

%%%%%%%%%%%%%%%%%%%%%%%%%%%%%%%%%%%%%%%%%%%%%%%
%                                             %
% CUSTOMISATION OF THE THESIS                 %
%                                             %
%%%%%%%%%%%%%%%%%%%%%%%%%%%%%%%%%%%%%%%%%%%%%%%

% Please customise your thesis with the macros below.

% TODO
\newcommand{\Firstnames}{Przemysław}
\newcommand{\Surname}{Baca}
\newcommand{\Supervisor}{Doctor of Engineering Anna Gorawska}  % supervisor (remove $\langle$ and $\rangle)
\newcommand{\Title}{Stock price prediction}
\newcommand{\TitleAlt}{Predykcja kursów walut}
\newcommand{\Program}{Control, Electronic, and Information Engineering} 
\newcommand{\Specialisation}{Informatics}
\newcommand{\Id}{294686} %  (remove $\langle$ and $\rangle)
\newcommand{\Departament}{Katedra Informatyki Stosowanej} % your supervisor's departament  (remove $\langle$ and $\rangle)


% If you have a consultant for your thesis, put their name below ...
\newcommand{\Consultant}{}  %  (remove $\langle$ and $\rangle)
% ... else leave the braces empty:
%\newcommand{\Consultant}{} % no consultant

% end of thesis customisation
%%%%%%%%%%%%%%%%%%%%%%%%%%%%%%%%%%%%%%%%%%

%%%%%%%%%%%%%%%%%%%%%%%%%%%%%%%%%%%%%%%%%%%%%%%
%                                             %
% END OF CUSTOMISATION                        %
%                                             %
%%%%%%%%%%%%%%%%%%%%%%%%%%%%%%%%%%%%%%%%%%%%%%%

%%%%%%%%%%%%%%%%%%%%%%%%%%%%%%%%%%%%%%%%  


%%%%%%%%%%%%%%%%%%%%%%%%%%%%%%%%%%%%%%%%%%%%%%%
%                                             %
%   PLEASE DO NOT MODIFY THE SETTINGS BELOW!  %
%                                             %
%%%%%%%%%%%%%%%%%%%%%%%%%%%%%%%%%%%%%%%%%%%%%%%



\documentclass[a4paper,twoside,12pt]{book}
\usepackage[utf8]{inputenc}                                      
\usepackage[T1]{fontenc}  
\usepackage{amsmath,amsfonts,amssymb,amsthm}
\usepackage[polish,british]{babel} 
\usepackage{indentfirst}



\usepackage{ifxetex}

\ifxetex
	\usepackage{fontspec}
	\defaultfontfeatures{Mapping=tex—text} % to support TeX conventions like ``——-''
	\usepackage{xunicode} % Unicode support for LaTeX character names (accents, European chars, etc)
	\usepackage{xltxtra} % Extra customizations for XeLaTeX
\else
	\usepackage{lmodern}
\fi



\usepackage[margin=2.5cm]{geometry}
\usepackage{graphicx} 
\usepackage{hyperref}
\usepackage{booktabs}
\usepackage{tikz}
\usepackage{pgfplots}
\usepackage{mathtools}
\usepackage{geometry}
\usepackage{subcaption}   % subfigures
\usepackage[page]{appendix} % toc,




\usepackage{csquotes}
\usepackage[natbib=true,backend=bibtex,maxbibnames=99]{biblatex}  % compilation of bibliography with BibTeX
%\usepackage[natbib=true,backend=biber,maxbibnames=99]{biblatex}  % compilation of bibliography with Biber
\bibliography{biblio}

\usepackage{ifmtarg}   % empty commands  

\usepackage{setspace}
\onehalfspacing


\frenchspacing



%%%% TODO LIST GENERATOR %%%%%%%%%

\usepackage{color}
\definecolor{brickred}      {cmyk}{0   , 0.89, 0.94, 0.28}

\makeatletter \newcommand \kslistofremarks{\section*{Remarks} \@starttoc{rks}}
  \newcommand\l@uwagas[2]
    {\par\noindent \textbf{#2:} %\parbox{10cm}
{#1}\par} \makeatother


\newcommand{\ksremark}[1]{%
{%\marginpar{\textdbend}
{\color{brickred}{[#1]}}}%
\addcontentsline{rks}{uwagas}{\protect{#1}}%
}










%%%%%%%%%%%%%% END OF TODO LIST GENERATOR %%%%%%%%%%%  

%%%%%%%%%%%% FANCY HEADERS %%%%%%%%%%%%%%%
% no capitalisation of headers
\usepackage{fancyhdr}
\pagestyle{fancy}
\fancyhf{}
\fancyhead[LO]{\nouppercase{\it\rightmark}}
\fancyhead[RE]{\nouppercase{\it\leftmark}}
\fancyhead[LE,RO]{\it\thepage}


\fancypagestyle{onlyPageNumbers}{%
   \fancyhf{} 
   \fancyhead[LE,RO]{\it\thepage}
}

\fancypagestyle{noNumbers}{%
   \fancyhf{} 
   \fancyhead[LE,RO]{}
}


\fancypagestyle{PageNumbersChapterTitles}{%
   \fancyhf{} 
   \fancyhead[LO]{\nouppercase{\Firstnames\ \Surname}}
   \fancyhead[RE]{\nouppercase{\leftmark}} 
   \fancyfoot[CE, CO]{\thepage}
}
 



%%%%%%%%%%%%%%%%%%%%%%%%%%%







\newcounter{pagesWithoutNumbers}

%%%%%%%%%%%%%%%%%%%%%%%%%%% 
\usepackage{xstring}
\usepackage{ifthen}
\newcommand{\printOpiekun}[1]{%		

    \StrLen{\Consultant}[\mystringlen]
    \ifthenelse{\mystringlen > 0}%
    {%
       {\large{\bfseries CONSULTANT}\par}
       
       {\large{\bfseries \Consultant}\par}
    }%
    {}
} 
%
%%%%%%%%%%%%%%%%%%%%%%%%%%%%%%%%%%%%%%%%%%%%%%
 
% Please do not modify the lines below!
\author{\Firstnames\ \Surname}
\newcommand{\Author}{\Firstnames\ \MakeUppercase{\Surname}}
\newcommand{\Type}{FINAL PROJECT}
\newcommand{\Faculty}{Faculty of Automatic Control, Electronics and Computer Science}
\newcommand{\Polsl}{Silesian University of Technology}
\newcommand{\Logo}{politechnika_sl_logo_bw_pion_en.pdf}
\newcommand{\LeftId}{Student identification number}
\newcommand{\LeftProgram}{Programme}
\newcommand{\LeftSpecialisation}{Specialisation}
\newcommand{\LeftSUPERVISOR}{SUPERVISOR}
\newcommand{\LeftDEPARTMENT}{DEPARTMENT}
%%%%%%%%%%%%%%%%%%%%%%%%%%%%%%%%%%%%%%%%%%%%%%

%%%%%%%%%%%%%%%%%%%%%%%%%%%%%%%%%%%%%%%%%%%%%%%
%                                             %
% END OF SETTINGS                             %
%                                             %
%%%%%%%%%%%%%%%%%%%%%%%%%%%%%%%%%%%%%%%%%%%%%%%




%%%%%%%%%%%%%%%%%%%%%%%%%%%%%%%%%%%%%%%%%%%%%%%
%                                             %
% MY PACKAGES, SETTINGS ETC.                  %
%                                             %
%%%%%%%%%%%%%%%%%%%%%%%%%%%%%%%%%%%%%%%%%%%%%%%

% Put your packages, macros, setting here.


 
%%%%%%%%%%%%%%%%%%%%%%%%%%%%%%%%%%%%%%%%%%%%%%%%%%%%%%%%%%%%%%%%%%%%%
% listings
% packages: listings or minted
% % % % % % % % % % % % % % % % % % % % % % % % % % % % % % % % % % % 

% package listings
\usepackage{listings}
\lstset{%
morekeywords={string,exception,std,vector},% add the keyword you need
language=C++,% C, Matlab, Python, SQL, TeX, XML, bash, ... – vide https://www.ctan.org/pkg/listings
commentstyle=\textit,%
identifierstyle=\textsf,%
keywordstyle=\sffamily\bfseries, %\texttt, %
%captionpos=b,%
tabsize=3,%
frame=lines,%
numbers=left,%
numberstyle=\tiny,%
numbersep=5pt,%
breaklines=true,%
%morekeywords={descriptor_gaussian,descriptor,partition,fcm_possibilistic,dataset,my_exception,exception,std,vector},%
escapeinside={@*}{*@},%
}

% % % % % % % % % % % % % % % % % % % % % % % % % % % % % % % % % % % 
% package minted
% \usepackage{minted}

% This package requires a special command line option in compilation
% pdflatex -shell-escape main.tex
% xelatex  -shell-escape main.tex

%%%%%%%%%%%%%%%%%%%%%%%%%%%%%%%%%%%%%%%%%%%%%%%%%%%%%%%%%%%%%%%%%%%%%



%%%%%%%%%%%%%%%%%%%%%%%%%%%%%%%%%%%%%%%%%%%%%%%
%                                             %
% END OF MY PACKAGES, SETTINGS ETC.           %
%                                             %
%%%%%%%%%%%%%%%%%%%%%%%%%%%%%%%%%%%%%%%%%%%%%%%


 
%%%%%%%%%%%%%%%%%%%%%%%%%%%%%%%%%%%%%%%%   


\begin{document}
%\kslistofremarks 


%%%%%%%%%%%%%%%%%%%%%%%%%%%%%%%%%%%%%%%%%%%%%%%
%                                             %
%    PLEASE DO NOT MODIFY THE TITLE PAGE!     %
%                                             %
%%%%%%%%%%%%%%%%%%%%%%%%%%%%%%%%%%%%%%%%%%%%%%%


%%%%%%%%%%%%%%%%%%  TITLE PAGE %%%%%%%%%%%%%%%%%%%
\pagestyle{empty}
{
	\newgeometry{top=1.5cm,%
	             bottom=2.5cm,%
	             left=3cm,
	             right=2.5cm}
 
	\ifxetex 
	  \begingroup
	  \setsansfont{Calibri}
	   
	\fi 
	 \sffamily
	\begin{center}
	\includegraphics[width=50mm]{\Logo}
	 
	
	{\Large\bfseries\Type\par}
	
	\vfill  \vfill  
			 
	{\large\Title\par}
	
	\vfill  
		
	{\large\bfseries\Author\par}
	
	{\normalsize\bfseries \LeftId: \Id}
	
	\vfill  		
 
	{\large{\bfseries \LeftProgram:} \Program\par} 
	
	{\large{\bfseries \LeftSpecialisation:} \Specialisation\par} 
	 		
	\vfill  \vfill 	\vfill 	\vfill 	\vfill 	\vfill 	\vfill  
	 
	{\large{\bfseries \LeftSUPERVISOR}\par}
	
	{\large{\bfseries \Supervisor}\par}
				
	{\large{\bfseries \LeftDEPARTMENT\ \Departament} \par}
		
	{\large{\bfseries \Faculty}\par}
		
	\vfill  \vfill  

    	
    \printOpiekun{\Consultant}
    
	\vfill  \vfill  
		
    {\large\bfseries  Gliwice \the\year}

   \end{center}	
       \ifxetex 
       	  \endgroup
       \fi
	\restoregeometry
}
  


\cleardoublepage
 
\rmfamily\normalfont
\pagestyle{empty}

  
%%% Let's start the thesis %%%%

% TODO
\subsubsection*{Stock price prediction} \Title

\subsubsection*{Abstract}  
(Thesis abstract – to be copied into an appropriate field during an electronic submission – in English.)

\subsubsection*{Keywords} 
(2-5 keywords, separated with commas)

\subsubsection*{Predykcja kursów walut} 
\begin{otherlanguage}{polish}
\TitleAlt
\end{otherlanguage}

\subsubsection*{Streszczenie} 
\begin{otherlanguage}{polish}
(Thesis abstract – to be copied into an appropriate field during an electronic submission – in Polish.)
\end{otherlanguage}
\subsubsection*{Słowa kluczowe}  
\begin{otherlanguage}{polish}
(2-5 keywords, separated by commas, in Polish)
\end{otherlanguage}




%%%%%%%%%%%%%%%%%% Table of contents %%%%%%%%%%%%%%%%%%%%%%
%\pagenumbering{Roman}
\thispagestyle{empty}
\tableofcontents
\thispagestyle{empty}

%%%%%%%%%%%%%%%%%%%%%%%%%%%%%%%%%%%%%%%%%%%%%%%%%%%%%
\setcounter{pagesWithoutNumbers}{\value{page}}
\mainmatter
\pagestyle{empty}
 
\cleardoublepage

\pagestyle{PageNumbersChapterTitles}

%%%%%%%%%%%%%% body of the thesis %%%%%%%%%%%%%%%%%

% TODO
\chapter{Introduction}

\begin{itemize}
	Recently, there has been significant progress in terms of Artificial Intelligence (later called AI) and Machine Learning (later called ML). Projects that in the past could barely recognize some simple objects or try to recognize some patterns are now surpassed by programs like advanced chat bots that can talk about literally anything or AI’s that can create very realistic art. Many famous personas like Elon Musk or Bill Gates start to recognize the potential of AI. 
	As this field is rapidly growing it appears to be both reasonable and demanded to perform research in this area.
	
\item introduction into the problem domain
\item settling of the problem in the domain
\item objective of the thesis 
\item scope of the thesis
\item short description of chapters
\item clear description of contribution of the thesis's author – in case of more authors table with enumeration of contribution of authors
\end{itemize}


% TODO
\chapter{[Problem analysis]}

\begin{itemize}
\item  problem analysis
\item state of the art, problem statement

As the Cambridge Dictionary [add to bibliography] states, Artificial Intelligence is “the study of how to produce machines that have some of the qualities that the human mind has, such as the ability to understand language, recognize pictures, solve problems, and learn”. It can be concluded that Artificial Intelligence is a field of science whose goal is to replace the work of humans with machines.
AI is used for a wide variety of tasks and applications. Some examples include:
Image and speech recognition: AI algorithms can be trained to identify objects, people, or words in images or audio recordings.
Language understanding: AI can be used to interpret and respond to natural language inputs, such as in virtual assistants or chatbots.
Predictive analytics: AI can be used to analyze large data sets and make predictions about future events or trends.
Robotics: AI controls and navigates robots, allowing them to perform tasks such as assembly line work or search-and-rescue operations.
Self-driving cars: AI is used to interpret sensor data and decide how to control the car.
Healthcare: AI can be used to analyze medical images, identify potential health risks, and assist doctors in making diagnoses.
Fraud detection: AI can be used to identify suspicious patterns of behavior in financial transactions.
These are just a few examples of the many ways AI is being used today, and as the technology continues to advance, it is likely to be used in an even wider range of applications in the future.

Since the beginning of Artificial Intelligence, people have wanted to use it for profit. The concept of a program that would be able to predict the future of the stock market is an idea that inspires many engineers to act. This is an unachievable goal, but AI is constantly evolving thanks to this challenge.

Predicting stock prices is a challenging task that has been the subject of much research in AI and machine learning. Various neural network architectures have been used for this task, and the choice of architecture will depend on the specific characteristics of the data and the problem at hand.
Some popular architectures that have been used for stock price prediction include:
Recurrent Neural Networks (RNNs): RNNs are particularly well-suited to sequential data, such as time series data, and have been used to predict stock prices by analyzing historical price and trading volume data.
Long Short-Term Memory (LSTM) networks: LSTMs are a type of RNN that are able to maintain a long-term memory of the inputs it has seen and has been used to predict stock prices by analyzing historical price data.
Convolutional Neural Networks (CNNs): CNNs are a type of neural network that are particularly well-suited to image and video data and have been used to predict stock prices by analyzing images of stock charts.
Gated Recurrent Units (GRUs): Similar to LSTMs but with less computational complexity.
It is worth noting that stock price prediction is a complex problem, and only some models would work best for some scenarios. Many other variables, such as news, announcements, market sentiments, and trends, also need to be taken into account and can be incorporated into the model using techniques like Sentiment Analysis and Named Entity Recognition.
It is also important to mention that stock prices are highly dynamic and non-linear, making it difficult to predict with high accuracy, and multiple models with different architectures may need to be combined to produce an accurate prediction.


For this project, LSTM model have been chosen to predict the stock prices. 
One of the main reasons why LSTMs may be well-suited for stock price prediction is their ability to maintain a long-term memory of the inputs they have seen. Stock prices are a type of time series data, which means that the current price of a stock is likely to be influenced by a long sequence of previous prices and other related financial data, such as trading volume and market sentiment. Because LSTMs can maintain a long-term memory of the inputs they have seen, they can understand patterns or dependencies in the data that may span multiple time steps, making them well-suited to analyze historical price data and make predictions about future price movements.
Another reason LSTMs may be well-suited for stock price prediction is that they can handle input data with high dimensionality and noise. Stock prices are influenced by various factors, such as economic indicators, company news, and global events, which can be challenging to quantify or predict. LSTMs are able to handle input data with high dimensionality and high noise, making them well-suited to analyze a wide range of financial data and make predictions about future price movements.
Stock price prediction is a complicated problem and no single model would work best for every scenario. LSTM might work well on one dataset and not on another. Also, LSTMs alone might not be enough to accurately predict stock prices, and other techniques such as Sentiment Analysis and Named Entity Recognition, as well as other variables such as news, announcements, market sentiments and trends, may need to be taken into account and can be incorporated into the model.


LSTM stands for Long Short-Term Memory. It is a type of Recurrent Neural Network (RNN) architecture that is used to process sequential data, such as time series or natural language.
An RNN processes input data one step at a time, maintaining an internal "memory" of the inputs it has seen. This allows the network to understand patterns or dependencies in the data that may span multiple time steps. However, the simple RNN architecture has a problem known as the "vanishing gradients" problem, where the gradients of the error concerning the parameters of the network become very small as they are backpropagated through multiple time steps.
By introducing a number of "gates" that regulate the movement of data into and out of the network's memory, LSTM finds a solution to this issue. These gates can be thought of as "switches" that turn the flow of information on and off, allowing the network to maintain a much longer-term memory of the inputs it has seen.
The LSTM architecture is used in various sequential data processing tasks, such as natural language processing, speech recognition, and time series prediction. LSTM is particularly well-suited to tasks where the output depends on a long sequence of prior inputs, such as in natural language processing or stock price prediction, as the output depends on many data

Artificial intelligence [Write what is an AI and what it is used for]
Recently, there has been significant progress in terms of Artificial Intelligence (later called AI) and Machine Learning (later called ML). Projects that in the past could barely recognize some simple objects or try to recognize some patterns are now surpassed by programs like advanced chat bots that can talk about literally anything or AI’s that can create very realistic art. Many famous personas like Elon Musk or Bill Gates start to recognize the potential of AI. 
As this field is rapidly growing it appears to be both reasonable and demanded to perform research in this area.

What is AI and what can it be used for
As the Cambridge Dictionary [add to bibliography] states, Artificial Intelligence is “the study of how to produce machines that have some of the qualities that the human mind has, such as the ability to understand language, recognize pictures, solve problems, and learn”. It can be concluded that Artificial Intelligence is a field of science whose goal is to replace the work of humans with machines.
AI is used for a wide variety of tasks and applications. Some examples include:
Image and speech recognition: AI algorithms can be trained to identify objects, people, or words in images or audio recordings.
Language understanding: AI can be used to interpret and respond to natural language inputs, such as in virtual assistants or chatbots.
Predictive analytics: AI can be used to analyze large data sets and make predictions about future events or trends.
Robotics: AI controls and navigates robots, allowing them to perform tasks such as assembly line work or search-and-rescue operations.
Self-driving cars: AI is used to interpret sensor data and decide how to control the car.
Healthcare: AI can be used to analyze medical images, identify potential health risks, and assist doctors in making diagnoses.
Fraud detection: AI can be used to identify suspicious patterns of behavior in financial transactions.
These are just a few examples of the many ways AI is being used today, and as the technology continues to advance, it is likely to be used in an even wider range of applications in the future.


AI in stock prediction
Since the beginning of Artificial Intelligence, people have wanted to use it for profit. The concept of a program that would be able to predict the future of the stock market is an idea that inspires many engineers to act. This is an unachievable goal, but AI is constantly evolving thanks to this challenge.

Predicting stock prices is a challenging task that has been the subject of much research in AI and machine learning. Various neural network architectures have been used for this task, and the choice of architecture will depend on the specific characteristics of the data and the problem at hand.
Some popular architectures that have been used for stock price prediction include:
Recurrent Neural Networks (RNNs): RNNs are particularly well-suited to sequential data, such as time series data, and have been used to predict stock prices by analyzing historical price and trading volume data.
Long Short-Term Memory (LSTM) networks: LSTMs are a type of RNN that are able to maintain a long-term memory of the inputs it has seen and has been used to predict stock prices by analyzing historical price data.
Convolutional Neural Networks (CNNs): CNNs are a type of neural network that are particularly well-suited to image and video data and have been used to predict stock prices by analyzing images of stock charts.
Gated Recurrent Units (GRUs): Similar to LSTMs but with less computational complexity.
It is worth noting that stock price prediction is a complex problem, and only some models would work best for some scenarios. Many other variables, such as news, announcements, market sentiments, and trends, also need to be taken into account and can be incorporated into the model using techniques like Sentiment Analysis and Named Entity Recognition.
It is also important to mention that stock prices are highly dynamic and non-linear, making it difficult to predict with high accuracy, and multiple models with different architectures may need to be combined to produce an accurate prediction.


For this project, LSTM model have been chosen to predict the stock prices. 
One of the main reasons why LSTMs may be well-suited for stock price prediction is their ability to maintain a long-term memory of the inputs they have seen. Stock prices are a type of time series data, which means that the current price of a stock is likely to be influenced by a long sequence of previous prices and other related financial data, such as trading volume and market sentiment. Because LSTMs can maintain a long-term memory of the inputs they have seen, they can understand patterns or dependencies in the data that may span multiple time steps, making them well-suited to analyze historical price data and make predictions about future price movements.
Another reason LSTMs may be well-suited for stock price prediction is that they can handle input data with high dimensionality and noise. Stock prices are influenced by various factors, such as economic indicators, company news, and global events, which can be challenging to quantify or predict. LSTMs are able to handle input data with high dimensionality and high noise, making them well-suited to analyze a wide range of financial data and make predictions about future price movements.
Stock price prediction is a complicated problem and no single model would work best for every scenario. LSTM might work well on one dataset and not on another. Also, LSTMs alone might not be enough to accurately predict stock prices, and other techniques such as Sentiment Analysis and Named Entity Recognition, as well as other variables such as news, announcements, market sentiments and trends, may need to be taken into account and can be incorporated into the model.


LSTM stands for Long Short-Term Memory. It is a type of Recurrent Neural Network (RNN) architecture that is used to process sequential data, such as time series or natural language.
An RNN processes input data one step at a time, maintaining an internal "memory" of the inputs it has seen. This allows the network to understand patterns or dependencies in the data that may span multiple time steps. However, the simple RNN architecture has a problem known as the "vanishing gradients" problem, where the gradients of the error concerning the parameters of the network become very small as they are backpropagated through multiple time steps.
By introducing a number of "gates" that regulate the movement of data into and out of the network's memory, LSTM finds a solution to this issue. These gates can be thought of as "switches" that turn the flow of information on and off, allowing the network to maintain a much longer-term memory of the inputs it has seen.
The LSTM architecture is used in various sequential data processing tasks, such as natural language processing, speech recognition, and time series prediction. LSTM is particularly well-suited to tasks where the output depends on a long sequence of prior inputs, such as in natural language processing or stock price prediction, as the output depends on many data

Prediction of the stock market
Stock prices can be predicted using a combination of methods, including technical analysis, fundamental analysis, quantitative analysis, and news analysis.
Technical analysis: This method involves analyzing charts and historical data to identify patterns and trends that can indicate future stock price movements. Technical analysts look at factors such as support and resistance levels, moving averages, and trading volume to make predictions about future price movements.
Fundamental analysis: This method involves analyzing a company's financial and economic fundamentals, such as earnings, revenue, and debt, to determine its intrinsic value and potential for future growth. Fundamental analysts look at a company's financial statements and other financial metrics to predict its future performance and stock price.
Quantitative analysis: This method involves using mathematical models and algorithms to analyze financial data and make predictions about future stock prices. Quantitative analysts use statistical methods and machine learning techniques to analyze large amounts of data and make predictions about future stock prices.
News analysis: This method involves analyzing news and events that may impact a company's stock price, such as changes in management, mergers, and acquisitions, or regulatory developments.
It is important to note that no method is perfect, and past performance does not guarantee future results. Additionally, stock prices can be affected by many factors, including macroeconomic conditions, geopolitical events, and investor sentiment, which can be challenging to predict.

\item  literature research (all sources in the thesis have to be referenced \cite{bib:article,bib:book,bib:conference,bib:internet})
\item description of existing solutions (also scientific ones, if the problem is scientifically researched), algorithms,  location of the thesis in the scientific domain
\end{itemize}



Mathematical formulae  
\begin{align}
y = \frac{\partial x}{\partial t}
\end{align}
and single math symbols $x$ and $y$ are typeset in the mathematical mode.



% TODO
\chapter{Requirements and tools}

\begin{itemize}

	Functional
	Functional Requirements for a Stock Price Prediction System
	Data Input: The system must be able to accept and process historical stock prices; depending on the choice of the user, it should be able to read open price, close price, highest price, lowest price, and average price.
	Data Preprocessing: The system must be able to clean, format, and prepare the data for analysis. It should scale the data appropriately to make it easier to fetch it into the neural network.
	Model Training: The system must be able to train an LSTM neural network model using historical data.
	Model Evaluation: The system must be able to evaluate the performance of the trained models.
	Stock Price Prediction: The system must be able to predict future stock prices based on the trained model. It should predict the prices based on the chosen parameters (company name).
	Data Visualization: The system must provide clear and intuitive visualizations of the predicted stock prices and historical data, including line charts and other relevant charts.
	Scalability: The system must be able to handle large amounts of data and be easily scalable to accommodate future growth in data and user base.
	
	
	Non functional
	Non-Functional Requirements for a Stock Price Prediction System
	Performance: The system must be able to predict stock prices with a response time of fewer than 10 seconds.
	Scalability: The system must be able to handle increasing amounts of data and user traffic without a significant decrease in performance.
	Reliability: The system must have a high level of reliability.
	Maintainability: The system must be easy to maintain, with regular updates and patches to fix bugs and improve performance.
	Compliance: The system must comply with relevant laws and regulations, such as data privacy laws and financial regulations.
	Monitorability: The system must provide monitoring and logging capabilities to track system usage and performance.
	Customizability: The system must be customizable according to user preferences and requirements.
	
	
	Chosen technology
	Python has been chosen for this project as it is a highly recommended language for developing and researching neural networks and AI of all kinds. It is both intuitive to use and offers powerful features when it comes to working with data. Many data scientists use python as it offers a wide range of libraries that make working with data very efficient. Python is also very popular and has a great community supporting this language.
	
	Keras has been chosen for deep learning for its many great features. Firstly, Keras is widely recommended for its simplicity. It helps with working with neural networks and is intuitive to use.
	Even though it is easy to use, it also offers a high level of abstraction so that the user can write their own methods and classes and experiment with different system architectures.
	What is more, Keras also has a large community, so it is easy to seek help and discuss possible solutions with other users.
	And lastly, even though Python has been chosen for this project, it is worth noting that Keras works with many languages and can be used by developers from many different fields.
	
	Yahoo Finance is a financial news and information website that provides a variety of services and tools for individuals and businesses. It offers real-time stock quotes, financial news, market data, and investment research and analysis. It also provides API for fast and easy communication between the site and the program that can be written. That is why it was chosen to ensure constant access to recent data that is fast at the same time.
	There are few python libraries that help with the reading data directly from Yahoo Finance. For this project, yahoofinancials has to be chosen as it provides easy access to raw data. Other libraries (e.g. yfinance or pandas_datareader) either offer too much processing of the data or need to be faster.
	
	NumPy is a Python library that stands for 'Numerical Python'. It is used for working with arrays of numerical data and provides functions for performing mathematical operations on these arrays. It is a fundamental library for scientific computing in Python.
	
	Pyplot is a sublibrary within the Matplotlib library for Python. It provides a convenient interface for creating a variety of different types of plots and charts, such as line plots, scatter plots, histograms, and more. Pyplot is based on the MATLAB plotting functions and provides a similar interface to the MATLAB plotting functions. It provides a simple way to create plots and charts with just a few lines of code.
	
	Scikit-learn (also known as sklearn) is a machine learning library for the Python programming language. It provides a range of supervised and unsupervised learning algorithms in Python for data mining and data analysis. It provides several preprocessing tools that can be used to prepare data for machine learning algorithms.
	
	from yahoofinancials import YahooFinancials
	import numpy as np
	import matplotlib.pyplot as plt
	from sklearn.preprocessing import minmax_scale
	from keras.models import Sequential
	from keras.layers import Dense, Dropout, LSTM
	

\item functional and nonfunctional requirements
\item use cases (UML diagrams)
\item description of tools
\item methodology of design and implementation
\end{itemize} 

% TODO
\chapter{External specification}
\begin{itemize}

	This section consists of variables that are used later in the project. By changing these values, modification of the application is possible. All variables that matter are stored here. company_name and price_type are used in reading data from Yahoo Finance. In company_name, the NASDAQ (National Association of Securities Dealers Automated Quotations) ticker symbol should be imputed so that the appropriate company data could be read. The ticker symbol is an abbreviation of the company name used in NASDAQ to associate a particular company with its data. In price_type, the type of price that should be read should be specified from open price (open), closing price (close), the highest price of the given period of time (high), lowest price of the given period of time (low) or the volume traded (volume).

	# Variables
	company_name = 'GOOGL' # company NASDAQ name
	price_type = 'open' # open, close, high, low, volume
	
	
	# Time interval
	startDate = '2021-06-01'
	endDate = '2022-06-01'
	interval = 'daily' # daily, weekly or monthly
	
	
	# Shape of input data
	chunkSize = 20
	Listing XXX Variables.
	
	YahooFinancials is easy to use and outputs raw data that can be conveniently used later. data_price_raw stores all data of a given company in the time period between startDate and endDate in chosen intervals. data_prices stores only the prices without the index of date, name, et cetera.
	Two ‘for loops’ append the data to the arrays created earlier. The data is assigned so that the neural network’s input data is the chosen number of days, and the test data is the next day so that the next day can be predicted on the chosen number of the days before it. For example, for chosen 30 days, the input data are 29 days, and the output should be the one next day (29:1).
	
	yahoo_financials = YahooFinancials(company_name)
	data_prices_raw = yahoo_financials.get_historical_price_data(startDate, endDate, interval)
	data_prices = data_prices_raw[company_name]['prices']
	
	
	for i in range(len(data_prices)):
	 data.append(data_prices[i][price_type])
	
	
	for i in range(chunkSize, len(data_prices) - 1):
	 data_prediction.append(data[i-chunkSize:i])
	 data_result.append(data[i])
	Listing xXX.
	
	In order to train the model, the gathered data should be reshaped so that they have three dimensions. This operation can be done with a few arithmetic operations, but it is more convenient to use reliable methods to do so. Firstly, the given data is transformed with minmax_scale() to normalize the data between 0 and 0.999. Then using the numpy reshape() function, the data is reshaped into three dimensions.
	
	# Reshape the data
	# into 3 dimensions
	data_result, data_prediction = np.array(data_result), np.array(data_prediction)
	
	
	data_prediction = minmax_scale(data_prediction, feature_range=(0,.999))
	data_prediction = np.reshape(data_prediction, (data_prediction.shape[0], data_prediction.shape[1], 1))
	data_result = minmax_scale(data_result, feature_range=(0,.999)
	Listing XXX.
	
	NOT CHECKED WITH GRAMMARLY FROM HERE
	As it was stated before in this paper, the chosen type of model for this project is Sequential model as it was proven to work the best for such solutions in different papers (pls link the papers here). In this project, several combinations of different layers have been considered and finally layers presented below have been added to the model. 
	
	LSTM -> Dropout -> LSTM -> Dropout -> Dense
	
	model = Sequential()
	model.add(LSTM(50, return_sequences=True, input_shape=(data_prediction.shape[1], 1)))
	model.add(Dropout(0.2))
	model.add(LSTM(50, return_sequences=False))
	model.add(Dropout(0.2))
	model.add(Dense(1))
	
	
	Model has been compiled using the ‘adam’ optimizer and for loss ‘mean squared error’ has been used.
	
	model.compile(optimizer='adam',
				  loss='mean_squared_error')
	
	
	PyPlot has been used for plotting the result for testing purposes as it provides very clear and easy to read graphs.. In order to verify how this model can predict the future prices the two graphs have been combined - the one with the real prices (blue) and the one with the predicted prices (red).

	

\item hardware and software requirements
\item installation procedure
\item activation procedure
\item types of users
\item user manual
\item system administration
\item security issues
\item example of usage
\item working scenarios (with screenshots or output files)
\end{itemize}



 
\begin{figure}
\centering
\begin{tikzpicture}
\begin{axis}[
    y tick label style={
        /pgf/number format/.cd,
            fixed,    
            fixed zerofill, % 1.0 zamiast 1
            precision=1,
        /tikz/.cd
    },
    x tick label style={
        /pgf/number format/.cd,
            fixed,
            fixed zerofill,
            precision=2,
        /tikz/.cd
    }
]
\addplot [domain=0.0:0.1] {rnd};
\end{axis} 
\end{tikzpicture}
\caption{Figure caption (below the figure).}
\label{fig:2}
\end{figure}

%%%%%%%%%%%%%%%%%%%%%
% FIGURE FROM FILE
%
%\begin{figure}
%\centering
%\includegraphics[width=0.5\textwidth]{./politechnika_sl_logo_bw_pion_en.pdf}
%\caption{Caption of a figure is always below the figure.}
%\label{fig:label}
%\end{figure}
%Fig. \ref{fig:label} presents …
%%%%%%%%%%%%%%%%%%%%%
%
%%%%%%%%%%%%%%%%%%%%
%% SUBFIGURES
%
%\begin{figure}
%\centering
%\begin{subfigure}{0.4\textwidth}
%    \includegraphics[width=\textwidth]{./politechnika_sl_logo_bw_pion_en.pdf}
%    \caption{Upper left figure.}
%    \label{fig:upper-left}
%\end{subfigure}
%\hfill
%\begin{subfigure}{0.4\textwidth}
%    \includegraphics[width=\textwidth]{./politechnika_sl_logo_bw_pion_en.pdf}
%    \caption{Upper right figure.}
%    \label{fig:upper-right}
%\end{subfigure}
%
%\begin{subfigure}{0.4\textwidth}
%    \includegraphics[width=\textwidth]{./politechnika_sl_logo_bw_pion_en.pdf}
%    \caption{Lower left figure.}
%    \label{fig:lower-left}
%\end{subfigure}
%\hfill
%\begin{subfigure}{0.4\textwidth}
%    \includegraphics[width=\textwidth]{./politechnika_sl_logo_bw_pion_en.pdf}
%    \caption{Lower right figure.}
%    \label{fig:lower-right}
%\end{subfigure}
%        
%\caption{Common caption for all subfigures.}
%\label{fig:subfigures}
%\end{figure}
%Fig. \ref{fig:subfigures} presents very important information, eg. Fig. \ref{fig:upper-right} is an upper right subfigure.
%%%%%%%%%%%%%%%%%%%%%


% TODO
\chapter{Internal specification}

\begin{itemize}
\item concept of the system
\item system architecture
\item description of data structures (and data bases)
\item components, modules, libraries, resume of important classes (if used)
\item resume of important algorithms (if used)
\item details of implementation of selected parts
\item applied design patterns
\item UML diagrams
\end{itemize}



% % % % % % % % % % % % % % % % % % % % % % % % % % % % % % % % % % % 
% To use the minted packages uncomment the package import in        %
% file config/settings.tex :  \usepackage{minted}                   %
% and compile with the shell escape                                 %
% pdflatex -shell-escape main                                       %
% % % % % % % % % % % % % % % % % % % % % % % % % % % % % % % % % % % 


Use special environments for inline code, eg  \lstinline|int a;| (package \texttt{listings})% or  \mintinline{C++}|int a;| (package \texttt{minted})
. Longer parts of code put in the figure environment, eg. code in Fig. \ref{fig:pseudocode:listings}% and Fig. \ref{fig:pseudocode:minted}
. Very long listings–move to an appendix.


\clearpage
\begin{figure}
\centering
\begin{lstlisting}
class test : public basic
{
    public:
      test (int a);
      friend std::ostream operator<<(std::ostream & s, 
                                     const test & t);
    protected:
      int _a;  
      
};
\end{lstlisting}
\caption{Pseudocode in \texttt{listings}.}
\label{fig:pseudocode:listings}
\end{figure}

%\begin{figure}
%\centering
%\begin{minted}[linenos,frame=lines]{c++}
%class test : public basic
%{
%    public:
%      test (int a);
%      friend std::ostream operator<<(std::ostream & s, 
%                                     const test & t);
%    protected:
%      int _a;  
%      
%};
%\end{minted}
%\caption{Pseudocode in \texttt{minted}.}
%\label{fig:pseudocode:minted}
%\end{figure}




% TODO
\chapter{Verification and validation}
\begin{itemize}
\item testing paradigm (eg V model)
\item test cases, testing scope (full / partial)
\item detected and fixed bugs
\item results of experiments (optional)
\end{itemize}

 
\begin{table}
\centering
\caption{A caption of a table is \textbf{above} it.}
\label{id:tab:wyniki}
\begin{tabular}{rrrrrrrr}
\toprule
	         &                                     \multicolumn{7}{c}{method}                                      \\
	         \cmidrule{2-8}
	         &         &         &        \multicolumn{3}{c}{alg. 3}        & \multicolumn{2}{c}{alg. 4, $\gamma = 2$} \\
	         \cmidrule(r){4-6}\cmidrule(r){7-8}
	$\zeta$ &     alg. 1 &   alg. 2 & $\alpha= 1.5$ & $\alpha= 2$ & $\alpha= 3$ &   $\beta = 0.1$  &   $\beta = -0.1$ \\
\midrule
	       0 &  8.3250 & 1.45305 &       7.5791 &    14.8517 &    20.0028 & 1.16396 &                       1.1365 \\
	       5 &  0.6111 & 2.27126 &       6.9952 &    13.8560 &    18.6064 & 1.18659 &                       1.1630 \\
	      10 & 11.6126 & 2.69218 &       6.2520 &    12.5202 &    16.8278 & 1.23180 &                       1.2045 \\
	      15 &  0.5665 & 2.95046 &       5.7753 &    11.4588 &    15.4837 & 1.25131 &                       1.2614 \\
	      20 & 15.8728 & 3.07225 &       5.3071 &    10.3935 &    13.8738 & 1.25307 &                       1.2217 \\
	      25 &  0.9791 & 3.19034 &       5.4575 &     9.9533 &    13.0721 & 1.27104 &                       1.2640 \\
	      30 &  2.0228 & 3.27474 &       5.7461 &     9.7164 &    12.2637 & 1.33404 &                       1.3209 \\
	      35 & 13.4210 & 3.36086 &       6.6735 &    10.0442 &    12.0270 & 1.35385 &                       1.3059 \\
	      40 & 13.2226 & 3.36420 &       7.7248 &    10.4495 &    12.0379 & 1.34919 &                       1.2768 \\
	      45 & 12.8445 & 3.47436 &       8.5539 &    10.8552 &    12.2773 & 1.42303 &                       1.4362 \\
	      50 & 12.9245 & 3.58228 &       9.2702 &    11.2183 &    12.3990 & 1.40922 &                       1.3724 \\
\bottomrule
\end{tabular}
\end{table}  



% TODO
\chapter{Conclusions}
\begin{itemize}
\item achieved results with regard to objectives of the thesis and requirements
\item path of further development (eg functional extension …)
\item encountered difficulties and problems
\end{itemize}



\backmatter 

%\bibliographystyle{plain}  % bibtex
%\bibliography{biblio} % bibtex
\printbibliography           % biblatex 
\addcontentsline{toc}{chapter}{Bibliography}

\begin{appendices}

% TODO

\chapter{Index of abbreviations and symbols}

\begin{itemize}
\item[DNA] deoxyribonucleic acid
\item[MVC] model--view--controller 
\item[$N$] cardinality of data set
\item[$\mu$] membership function of a fuzzy set
\item[$\mathbb{E}$] set of edges of a graph
\item[$\mathcal{L}$] Laplace transformation
\end{itemize}


% TODO
\chapter{Listings}

(Put long listings here.)

\begin{lstlisting}
if (_nClusters < 1)
	throw std::string ("unknown number of clusters");
if (_nIterations < 1 and _epsilon < 0)
	throw std::string ("You should set a maximal number of iteration or minimal difference -- epsilon.");
if (_nIterations > 0 and _epsilon > 0)
	throw std::string ("Both number of iterations and minimal epsilon set -- you should set either number of iterations or minimal epsilon.");
\end{lstlisting}


% % % % % % % % % % % % % % % % % % % % % % % % % % % % % % % % % % % 
% To use the minted packages uncomment the package import in        %
% file config/settings.tex :  \usepackage{minted}                   %
% and compile with the shell escape                                 %
% pdflatex -shell-escape main                                       %
% % % % % % % % % % % % % % % % % % % % % % % % % % % % % % % % % % % 

%\begin{minted}[linenos,breaklines,frame=lines]{c++}
%if (_nClusters < 1)
%	throw std::string ("unknown number of clusters");
%if (_nIterations < 1 and _epsilon < 0)
%	throw std::string ("You should set a maximal number of iteration or minimal difference -- epsilon.");
%if (_nIterations > 0 and _epsilon > 0)
%	throw std::string ("Both number of iterations and minimal epsilon set -- you should set either number of iterations or minimal epsilon.");
%\end{minted} 

% TODO
\chapter{List of additional files in~electronic submission (if applicable)}


Additional files uploaded to the system include:
\begin{itemize}
\item source code of the application,
\item test data,
\item a video file showing how software or hardware developed for thesis is used,
\item etc.
\end{itemize}

\listoffigures
\addcontentsline{toc}{chapter}{List of figures}
\listoftables
\addcontentsline{toc}{chapter}{List of tables}
	
\end{appendices}

\end{document}


%% Finis coronat opus.

