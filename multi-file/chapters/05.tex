\chapter{Internal specification}

\begin{itemize}
\item concept of the system
\item system architecture
\item description of data structures (and data bases)
\item components, modules, libraries, resume of important classes (if used)
\item resume of important algorithms (if used)
\item details of implementation of selected parts
\item applied design patterns
\item UML diagrams
\end{itemize}

\section*{Documentation}

This section consists of variables that are used later in the project. By changing these values, modification of the application is possible. All variables that matter are stored here. company{\_}name and price{\_}type are used in reading data from Yahoo Finance. In company{\_}name, the NASDAQ (National Association of Securities Dealers Automated Quotations) ticker symbol should be imputed so that the appropriate company data could be read. The ticker symbol is an abbreviation of the company name used in NASDAQ to associate a particular company with its data. In price{\_}type, the type of price that should be read should be specified from open price (open), closing price (close), the highest price of the given period of time (high), lowest price of the given period of time (low) or the volume traded (volume).

\clearpage
\begin{figure}
\centering
\begin{lstlisting}
  # Variables
  company_name = 'GOOGL' # company NASDAQ name
  price_type = 'open' # open, close, high, low, volume
  
  
  # Time interval
  startDate = '2021-06-01'
  endDate = '2022-06-01'
  interval = 'daily' # daily, weekly or monthly
  
  
  # Shape of input data
  chunkSize = 20
\end{lstlisting}
\caption{Variables.}
\label{fig:pseudocode:listings}
\end{figure}

YahooFinancials is easy to use and outputs raw data that can be conveniently used later. data{\_}price{\_}raw stores all data of a given company in the time period between startDate and endDate in chosen intervals. data{\_}prices stores only the prices without the index of date, name, et cetera.
Two ‘for loops’ append the data to the arrays created earlier. The data is assigned so that the neural network’s input data is the chosen number of days, and the test data is the next day so that the next day can be predicted on the chosen number of the days before it. For example, for chosen 30 days, the input data are 29 days, and the output should be the one next day (29:1).

\clearpage
\begin{figure}
\centering
\begin{lstlisting}
  yahoo_financials = YahooFinancials(company_name)
  data_prices_raw = yahoo_financials.get_historical_price_data(startDate, endDate, interval)
  data_prices = data_prices_raw[company_name]['prices']
  
  
  for i in range(len(data_prices)):
   data.append(data_prices[i][price_type])
  
  
  for i in range(chunkSize, len(data_prices) - 1):
   data_prediction.append(data[i-chunkSize:i])
   data_result.append(data[i])
\end{lstlisting}
\caption{Working with the stock market data.}
\label{fig:pseudocode:listings}
\end{figure}

In order to train the model, the gathered data should be reshaped so that they have three dimensions. This operation can be done with a few arithmetic operations, but it is more convenient to use reliable methods to do so. Firstly, the given data is transformed with minmax{\_}scale() to normalize the data between 0 and 0.999. Then using the numpy reshape() function, the data is reshaped into three dimensions.

\clearpage
\begin{figure}
\centering
\begin{lstlisting}
  # Reshape the data
  # into 3 dimensions
  data_result, data_prediction = np.array(data_result), np.array(data_prediction)
  
  
  data_prediction = minmax_scale(data_prediction, feature_range=(0,.999))
  data_prediction = np.reshape(data_prediction, (data_prediction.shape[0], data_prediction.shape[1], 1))
  data_result = minmax_scale(data_result, feature_range=(0,.999)
\end{lstlisting}
\caption{Reshaping data.}
\label{fig:pseudocode:listings}
\end{figure}