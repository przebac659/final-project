\subsubsection*{Thesis title} \Title

\subsubsection*{Abstract}  
The work focuses on the educational aspects of predicting stock prices on the American stock exchange using artificial intelligence and, more precisely, neural networks. The first part of the work discusses state of the art, that is, techniques currently used in predicting stock market behavior, various types of neural networks used in this task, and techniques used by financial analysts. Then, the work focuses on an application that allows experimenting with a Long short-term memory neural network. Firstly, the requirements and technologies used are discussed, and then the project documentation and user manual are presented. Finally, there is a graphical and analytical comparison of the different neural network models. The comparison shows that a different model may seem the best depending on the expected results. Finally, a plan for the future development of the application was presented.

\subsubsection*{Keywords} 
Stock market, Artificial Intelligence

\subsubsection*{Tytuł pracy} 
\begin{otherlanguage}{polish}
\TitleAlt
\end{otherlanguage}

\subsubsection*{Streszczenie} 
\begin{otherlanguage}{polish}
    Praca skupia się na edukacyjnych aspektach predykcji cen akcji na giełdzie amerykańskiej przy pomocy sztucznej inteligencji, a dokładniej sieci neuronowych. Pierwsza część pracy omawia obecne dokonania w tej dziedzinie (state of the art), czyli techniki aktualnie wykorzystywane w przewidywaniu zachowań giełdy, różne rodzaje sieci neuronowych używanych w tym zadaniu oraz techniki wykorzystywane przez analityków finansowych. Następnie praca skupia się na aplikacji umożliwiającej eksperymentowanie z siecią neuronową typu Long short-term memory. Najpierw są omawiane wymagania oraz użyte technologie, a następnie zaprezentowana jest dokumentacja projektu oraz instrukcja użytkownika. Na końcu znajduje się porównanie zarówno graficzne, jak i analityczne różnych modeli. Z porównania wynika, że w zależności od oczekiwanych rezultatów inny model może wydawać się najlepszy. Na końcu przedstawiono jeszcze plan przyszłego rozwoju aplikacji.
\end{otherlanguage}
\subsubsection*{Słowa kluczowe}  
\begin{otherlanguage}{polish}
Giełda Papierów Wartościowych, Sztuczna inteligencja
\end{otherlanguage}

