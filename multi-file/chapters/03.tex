\chapter{Requirements and tools}

\begin{itemize}
\item functional and nonfunctional requirements
\item use cases (UML diagrams)
\item description of tools
\item methodology of design and implementation
\end{itemize} 

\section*{Functional requirements}
Functional Requirements for a Stock Price Prediction System
\begin{itemize}
    \item Data Input: The system must be able to accept and process historical stock prices; depending on the choice of the user, it should be able to read open price, close price, highest price, lowest price, and average price.
    \item Data Preprocessing: The system must be able to clean, format, and prepare the data for analysis. It should scale the data appropriately to make it easier to fetch it into the neural network.
    \item Model Training: The system must be able to train an LSTM neural network model using historical data.
    \item Model Evaluation: The system must be able to evaluate the performance of the trained models.
    \item Stock Price Prediction: The system must be able to predict future stock prices based on the trained model. It should predict the prices based on the chosen parameters (company name).
    \item Data Visualization: The system must provide clear and intuitive visualizations of the predicted stock prices and historical data, including line charts and other relevant charts.
    \item Scalability: The system must be able to handle large amounts of data and be easily scalable to accommodate future growth in data and user base.
\end{itemize} 

\section*{Non-functional requirements}
Non-Functional Requirements for a Stock Price Prediction System
\begin{itemize}
    \item Performance: The system must be able to predict stock prices with a response time of fewer than 10 seconds.
    \item Scalability: The system must be able to handle increasing amounts of data and user traffic without a significant decrease in performance.
    \item Reliability: The system must have a high level of reliability.
    \item Maintainability: The system must be easy to maintain, with regular updates and patches to fix bugs and improve performance.
    \item Compliance: The system must comply with relevant laws and regulations, such as data privacy laws and financial regulations.
    \item Monitorability: The system must provide monitoring and logging capabilities to track system usage and performance.
    \item Customizability: The system must be customizable according to user preferences and requirements.
\end{itemize} 

\section*{Chosen technology}
Python has been chosen for this project as it is a highly recommended language for developing and researching neural networks and AI of all kinds. It is both intuitive to use and offers powerful features when it comes to working with data. Many data scientists use python as it offers a wide range of libraries that make working with data very efficient. Python is also very popular and has a great community supporting this language.
\par
Keras has been chosen for deep learning for its many great features. Firstly, Keras is widely recommended for its simplicity. It helps with working with neural networks and is intuitive to use.
Even though it is easy to use, it also offers a high level of abstraction so that the user can write their own methods and classes and experiment with different system architectures.
What is more, Keras also has a large community, so it is easy to seek help and discuss possible solutions with other users.
And lastly, even though Python has been chosen for this project, it is worth noting that Keras works with many languages and can be used by developers from many different fields.
\par
Yahoo Finance is a financial news and information website that provides a variety of services and tools for individuals and businesses. It offers real-time stock quotes, financial news, market data, and investment research and analysis. It also provides API for fast and easy communication between the site and the program that can be written. That is why it was chosen to ensure constant access to recent data that is fast at the same time.
There are few python libraries that help with the reading data directly from Yahoo Finance. For this project, yahoofinancials has to be chosen as it provides easy access to raw data. Other libraries (e.g. yfinance or pandas{\_}datareader) either offer too much processing of the data or need to be faster.
\par
NumPy is a Python library that stands for 'Numerical Python'. It is used for working with arrays of numerical data and provides functions for performing mathematical operations on these arrays. It is a fundamental library for scientific computing in Python.
\par
Pyplot is a sublibrary within the Matplotlib library for Python. It provides a convenient interface for creating a variety of different types of plots and charts, such as line plots, scatter plots, histograms, and more. Pyplot is based on the MATLAB plotting functions and provides a similar interface to the MATLAB plotting functions. It provides a simple way to create plots and charts with just a few lines of code.
\par
Scikit-learn (also known as sklearn) is a machine learning library for the Python programming language. It provides a range of supervised and unsupervised learning algorithms in Python for data mining and data analysis. It provides several preprocessing tools that can be used to prepare data for machine learning algorithms.
\par
PyQt is a framework dedicated to building applications with a graphical interface using Python programming language. PyQt provides a range of modules and classes that developers can use to create GUI elements, such as buttons, dialogs, menus, and more. It also provides support for handling events, managing layouts, and working with various data types and formats.
\clearpage
\begin{figure}
\centering
\begin{lstlisting}
    from PyQt6 import QtGui
    from PyQt6.QtGui import QPainter, QAction, QColor, QPixmap
    from PyQt6.QtWidgets import (
        QApplication,
        QMainWindow,
        QPushButton,
        QMenuBar,
        QMenu,
        QFrame,
        QGridLayout,
        QVBoxLayout,
        QHBoxLayout,
        QWidget,
        QSlider,
        QLabel,
        QComboBox,
    )
    
    from yahoofinancials import YahooFinancials
    
    from keras.models import Sequential, load_model
    from keras.layers import Dense, Dropout, LSTM
    
    from sys import argv
    
    import matplotlib.pyplot as plt
    
    import numpy as np
    from sklearn.preprocessing import minmax_scale
\end{lstlisting}
\caption{Libraries used for this project.}
\label{fig:pseudocode:listings}
\end{figure}

